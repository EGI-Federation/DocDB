\documentclass[12pt]{article}

\input{memo_setup}
\input{abbrev}
\usepackage{ewv_environ}      %   %% Define some new environements                            

%\graphicspath{{sigmac_0999/}}

\begin{document}                                                 

\title{BTeV Document Database Design and Interface}

\author{
        Eric W. Vaandering\\  
        Vanderbilt University\\
        \\
        {\small\url{http://www.hep.vanderbilt.edu/~somewhere}}
       }
        

\date{27 November 2001 \\
      Revised: 28 November 2001}

\maketitle

\begin{abstract} 
We describe the design and implementation of the BTeV document
database.\end{abstract}

\section{Design considerations}

We wanted a document database that satisfies a number of requirements:
\begin{itemize}
\item{Multiple revisions per document, old versions still available}
\item{Multiple files per document. Multiple file types (source and 
presentable) and/or child files (especially useful for trees of HTML 
documents)}
\item{All documents are kept on the BTeV web server, not local copies 
so we don't suffer from link-rot}
\item{The ability to upload documents by browser upload or forcing the 
document database to download via http.}
\end{itemize}

There are several security issues we want to resolve too:
\begin{itemize}
\item{The ability to not only have public/private documents but also 
documents that are accessible to subgroups (like RTES) which don't know 
the BTeV password. We also want to have documents that are only 
accessible to sub-groups (like the executive council).}
\item{Users should only know that a document exists if they have 
privileges to view it.}
\item{The ability to easily move documents, or just certain versions of 
documents, from restricted to public and vice-versa.}
\end{itemize}

\section{Interface}

The document database is accessible via web forms and CGI scripts which 
handle all accesses to the underlying database engines. Currently only 
the scripts to reserve a document number or to place the initial 
version of a document in the database are in place. Additionally, these 
scripts only support browser upload of a single file. They need to be 
extended to support http download and multiple files.

These scripts are all written in Perl using the Perl::CGI package and 
the Perl::DBI database access modules. There is no reason certain 
additional scripts can't be written using PHP, C++, or any other 
language. 

\section{Database}

The database design is believed to be basically finished. The document 
database is implemented as a number of different tables. Because each 
document can have multiple revisions and each revision can have 
multiple files (and authors and topics), there are three main tables 
Document, DocumentRevision, and DocumentFile. The hierarchy of these 
tables, their elements, and the auxiliary tables are shown in figref.

This database is implemented using MySQL. Currently the server is 
running on \url{vuhepv.phy.vanderbilt.edu}.  

\section{File system and security}

All documents are placed into a single directory tree. There is no 
division based on security or topic since these are fluid designations 
(and each document can have multiple values for these settings). The 
file system is arranged only by document number. Each revision of each 
document is placed in its own directory, which means that most 
directories will only have a small number of files. However, the 
advantage of this scheme is two-fold. First, each of the files of a 
particular revision need not be renamed. Second, access to each 
revision can be controlled with a potentially unique .htaccess file.

To avoid possible issues with large numbers of files in a single 
directory, the document directories are further divided into groups of 
100. For example, imagine that the 3rd revision of the 1029th document 
contains one file, text.ps. Then the file tree looks like this:

\texttt{\$DOCROOT/0010/001029/003/text.ps}

As you can see, the file system is designed for up to 999,999 documents 
each of which can have up to 999 versions.

\end{document}




